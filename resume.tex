\documentclass{resume}
\usepackage{zh_CN-Adobefonts_external} 
\usepackage{linespacing_fix}
\usepackage{cite}
\begin{document}
\pagenumbering{gobble}



%***"%"后面的所有内容是注释而非代码,不会输出到最后的PDF中
%***使用本模板,只需要参照输出的PDF,在本文档的相应位置做简单替换即可
%***修改之后,输出更新后的PDF,只需要点击Overleaf中的“Recompile”按钮即可
%**********************************姓名********************************************
\name{***}
%**********************************联系信息****************************************
%第一个括号里写手机号,第二个写邮箱
\contactInfo{(+86) ***********}{***}{}{}
%**********************************其他信息****************************************
%在大括号内填写其他信息,最多填写4个,但是如果选择不填信息,
%那么大括号必须空着不写,而不能删除大括号。

%后面的四个大括号里的所有信息都会在一行输出
%如果想要写两行,那就用两次这个指令(\otherInfo{}{}{}{})即可
\otherInfo{***}{GitHub: BH3GEI}{Blog: ***}{}

\otherInfo{Firmware Integration Intern, 12V & Body ControlFirmware Integration Intern, 12V & Body Control}{}{}{}
\otherInfo{}{}{}{}
%*********************************照片**********************************************
%照片需要放到images文件夹下,名字必须是you.jpg,如果不需要照片可以不添加此行命令
%0.15的意思是,照片的宽度是页面宽度的0.15倍,调整大小,避免遮挡文字
\yourphoto{0.10}
%**********************************正文**********************************************


%***大标题,下面有横线做分割
%***一般的标题有:教育背景,实习(项目)经历,工作经历,自我评价,求职意向,等等


%***********一行子标题**************
%***第一个大括号里的内容向左对齐,第二个大括号里的内容向右对齐
%***\textbf{}括号里的字是粗体,\textit{}括号里的字是斜体


\section{个人介绍}
\datedsubsection{\textbf{吉林大学},\textit{本科 | 应用物理学}}{2019.09 - 2023.06}


\begin{itemize}[parsep=0.5ex]
  \item 
  \item 
  
  \item 
  \item 

\end{itemize}




\section{项目经历}

\datedsubsection{\textbf{吉林大学吉甲大师 | TARS\_Go},电控组成员}{2020.7-2020.10}

\begin{itemize}[parsep=0.5ex]
  \item 
  \item 
\end{itemize}

\datedsubsection{\textbf{},核心成员}{2021.3-2022.4}
。
\begin{itemize}[parsep=0.5ex]
  \item 

  \item 
\end{itemize}

\datedsubsection{\textbf{},后端工程师}{2021.4-2021.9}



\begin{itemize}[parsep=0.5ex]
  \item 
  \item 
\end{itemize}


\datedsubsection{\textbf{ }}{2021.7-2022.6}
\begin{itemize}[parsep=0.5ex]
  \item 

  \item 
  \item ,\\
  
\end{itemize}



\section{竞赛与证书}

\begin{itemize}[parsep=0.5ex]
  \item  


\end{itemize}








\end{document}
\end{document}
